%% Generated by Sphinx.
\def\sphinxdocclass{report}
\documentclass[letterpaper,10pt,brazil]{sphinxmanual}
\ifdefined\pdfpxdimen
   \let\sphinxpxdimen\pdfpxdimen\else\newdimen\sphinxpxdimen
\fi \sphinxpxdimen=.75bp\relax
\ifdefined\pdfimageresolution
    \pdfimageresolution= \numexpr \dimexpr1in\relax/\sphinxpxdimen\relax
\fi
%% let collapsible pdf bookmarks panel have high depth per default
\PassOptionsToPackage{bookmarksdepth=5}{hyperref}

\PassOptionsToPackage{booktabs}{sphinx}
\PassOptionsToPackage{colorrows}{sphinx}

\PassOptionsToPackage{warn}{textcomp}
\usepackage[utf8]{inputenc}
\ifdefined\DeclareUnicodeCharacter
% support both utf8 and utf8x syntaxes
  \ifdefined\DeclareUnicodeCharacterAsOptional
    \def\sphinxDUC#1{\DeclareUnicodeCharacter{"#1}}
  \else
    \let\sphinxDUC\DeclareUnicodeCharacter
  \fi
  \sphinxDUC{00A0}{\nobreakspace}
  \sphinxDUC{2500}{\sphinxunichar{2500}}
  \sphinxDUC{2502}{\sphinxunichar{2502}}
  \sphinxDUC{2514}{\sphinxunichar{2514}}
  \sphinxDUC{251C}{\sphinxunichar{251C}}
  \sphinxDUC{2572}{\textbackslash}
\fi
\usepackage{cmap}
\usepackage[T1]{fontenc}
\usepackage{amsmath,amssymb,amstext}
\usepackage{babel}



\usepackage{tgtermes}
\usepackage{tgheros}
\renewcommand{\ttdefault}{txtt}



\usepackage[Sonny]{fncychap}
\ChNameVar{\Large\normalfont\sffamily}
\ChTitleVar{\Large\normalfont\sffamily}
\usepackage{sphinx}

\fvset{fontsize=auto}
\usepackage{geometry}


% Include hyperref last.
\usepackage{hyperref}
% Fix anchor placement for figures with captions.
\usepackage{hypcap}% it must be loaded after hyperref.
% Set up styles of URL: it should be placed after hyperref.
\urlstyle{same}

\addto\captionsbrazil{\renewcommand{\contentsname}{Contents:}}

\usepackage{sphinxmessages}
\setcounter{tocdepth}{1}



\title{3PST\_2.0.0}
\date{07 mar. 2024}
\release{2.0.0}
\author{INPE/COGPI/SEPEC}
\newcommand{\sphinxlogo}{\vbox{}}
\renewcommand{\releasename}{Release}
\makeindex
\begin{document}

\ifdefined\shorthandoff
  \ifnum\catcode`\=\string=\active\shorthandoff{=}\fi
  \ifnum\catcode`\"=\active\shorthandoff{"}\fi
\fi

\pagestyle{empty}
\sphinxmaketitle
\pagestyle{plain}
\sphinxtableofcontents
\pagestyle{normal}
\phantomsection\label{\detokenize{index::doc}}


\sphinxhref{https://www.gov.br/inpe/pt-br}{\sphinxincludegraphics{{../3pst2.0/Doc_3PST/_build/html/_static/img/inpelogo}.png}}

\sphinxAtStartPar
3PST 2.0.0 \textendash{} Sistema web para gestão do portfólio de
Programas, Projetos, Produtos, Serviços e Tecnologias do INPE.


\chapter{Partes interessadas}
\label{\detokenize{index:partes-interessadas}}\begin{itemize}
\item {} 
\sphinxAtStartPar
Funcionários do SEPEC: funcionários do setor, que participam diretamente da gestão dos dados da carteira;

\item {} 
\sphinxAtStartPar
Colaboradores da COGPI: colaboradores da coordenação, que participam indiretamente na gestão dos dados da carteira;

\item {} 
\sphinxAtStartPar
Funcionários do INPE: funcionários da instituição que pretendem abrir TAPs, TAPgs e TAS.

\end{itemize}


\chapter{Documentos}
\label{\detokenize{index:documentos}}\index{habilitarInput() (função interna)@\spxentry{habilitarInput()}\spxextra{função interna}}

\begin{fulllineitems}
\phantomsection\label{\detokenize{index:habilitarInput}}
\pysigstartsignatures
\pysiglinewithargsret{\sphinxbfcode{\sphinxupquote{\DUrole{n}{habilitarInput}}}}{\sphinxparam{\DUrole{n}{elemenId}}}{}
\pysigstopsignatures
\sphinxAtStartPar
A função \sphinxtitleref{habilitarInput()} habilita todos os campos de entrada de um formulário e
desabilita o botão “Alterar” ao ativar o botão “Salvar”.
\begin{quote}\begin{description}
\sphinxlineitem{Argumentos}\begin{itemize}
\item {} 
\sphinxAtStartPar
\sphinxstyleliteralstrong{\sphinxupquote{elemenId}} (\sphinxstyleliteralemphasis{\sphinxupquote{string}}) \textendash{} Retorna uma referência ao primeiro objeto com o valor especificado do atributo ID.

\end{itemize}

\end{description}\end{quote}

\end{fulllineitems}

\index{downloadCSVFile() (função interna)@\spxentry{downloadCSVFile()}\spxextra{função interna}}

\begin{fulllineitems}
\phantomsection\label{\detokenize{index:downloadCSVFile}}
\pysigstartsignatures
\pysiglinewithargsret{\sphinxbfcode{\sphinxupquote{\DUrole{n}{downloadCSVFile}}}}{\sphinxparam{\DUrole{n}{csv}}\sphinxparamcomma \sphinxparam{\DUrole{n}{filename}}\sphinxparamcomma \sphinxparam{\DUrole{n}{tagname}}}{}
\pysigstopsignatures
\sphinxAtStartPar
A função \sphinxtitleref{downloadCSVFile} é responsável por criar um arquivo CSV e iniciar seu download.
\begin{quote}\begin{description}
\sphinxlineitem{Argumentos}\begin{itemize}
\item {} 
\sphinxAtStartPar
\sphinxstyleliteralstrong{\sphinxupquote{csv}} (\sphinxstyleliteralemphasis{\sphinxupquote{string}}) \textendash{} O parâmetro \sphinxtitleref{csv} é uma string que representa o conteúdo do arquivo CSV que você deseja baixar.

\item {} 
\sphinxAtStartPar
\sphinxstyleliteralstrong{\sphinxupquote{filename}} (\sphinxstyleliteralemphasis{\sphinxupquote{string}}) \textendash{} O parâmetro ‘filename’ é uma string que especifica o nome do arquivo a ser baixado.

\item {} 
\sphinxAtStartPar
\sphinxstyleliteralstrong{\sphinxupquote{tagname}} (\sphinxstyleliteralemphasis{\sphinxupquote{string}}) \textendash{} Cria uma instância do elemento para a tag especificada.

\end{itemize}

\end{description}\end{quote}

\end{fulllineitems}

\index{htmlToCsv() (função interna)@\spxentry{htmlToCsv()}\spxextra{função interna}}

\begin{fulllineitems}
\phantomsection\label{\detokenize{index:htmlToCsv}}
\pysigstartsignatures
\pysiglinewithargsret{\sphinxbfcode{\sphinxupquote{\DUrole{n}{htmlToCsv}}}}{\sphinxparam{\DUrole{n}{filename}}}{}
\pysigstopsignatures
\sphinxAtStartPar
A função \sphinxtitleref{htmlToCsv} converte uma tabela HTML em um arquivo CSV e faz o download.
\begin{quote}\begin{description}
\sphinxlineitem{Argumentos}\begin{itemize}
\item {} 
\sphinxAtStartPar
\sphinxstyleliteralstrong{\sphinxupquote{filename}} (\sphinxstyleliteralemphasis{\sphinxupquote{string}}) \textendash{} O parâmetro \sphinxtitleref{filename} é uma string que representa o nome do arquivo CSV que será baixado.

\end{itemize}

\end{description}\end{quote}

\end{fulllineitems}



\chapter{Índices e tabelas}
\label{\detokenize{index:indices-e-tabelas}}\begin{itemize}
\item {} 
\sphinxAtStartPar
\DUrole{xref,std,std-ref}{genindex}

\item {} 
\sphinxAtStartPar
\DUrole{xref,std,std-ref}{modindex}

\item {} 
\sphinxAtStartPar
\DUrole{xref,std,std-ref}{search}

\end{itemize}



\renewcommand{\indexname}{Índice}
\printindex
\end{document}